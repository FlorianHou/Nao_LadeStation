\documentclass{report}
\usepackage{tikz}
\usepackage{ngerman}
\usetikzlibrary{shapes, arrows}

\title{Ablauf Ver-1.0}
\author{Zhaoyang Hou}
\date{\today}

\begin{document}
	\maketitle
	\pagestyle{empty}
	% Grund Formung
	\tikzstyle{startstop} = [rectangle, rounded corners, minimum width=3cm, minimum height=1cm,text centered, draw=black, fill=red!30]
	\tikzstyle{io} = [trapezium, trapezium left angle=70, trapezium right angle=110, minimum width=3cm, minimum height=1cm, text centered, draw=black, fill=blue!30]
	\tikzstyle{process} = [rectangle, minimum width=3cm, minimum height=1cm, text centered, draw=black, fill=orange!30]
	\tikzstyle{entscheidung} = [diamond, minimum width=3cm, minimum height=1cm, text centered, draw=black, fill=green!30]
	\tikzstyle{arrow} = [thick,->,>=stealth]
	
	\begin{tikzpicture}[node distance=1.6cm]
	%Boxs detalieren
	\node (start) [startstop] {Start};
	\node (pro1) [process, below of=start] {Bewegen zu Haupteingang};
	\node (pro2) [process, below of=pro1] {Stationen Nr. Erkennen};
	\node (ent1) [entscheidung, below of=pro2,yshift=-0.3cm] {Frei?};
	\node (pro2-2) [process, right of = ent1, xshift=2.5cm] {Nächst};
	\node (pro3) [process, below of=ent1,yshift=-0.3cm] {Suche Landmarkers für Zentierung};
	\node (pro4) [process, below of=pro3] {Information der Landmarks bearbeiten};
	\node (pro5) [process, below of=pro4] {Bewegt in y Richtung zu zentieren};
	\node (ent2) [entscheidung, below of=pro5,yshift=-0.5cm] {Centiert?};
	\node (pro6) [process, below of=ent2,yshift=-0.6cm] {Bewegt in x Richtung zu nähren};
	\node (pro7) [process, below of=pro6] {?????,nicht sicher,\\ weil LadeStation noch nicht festgelegt wird};
	\node (ende) [startstop, below of=pro7] {Ende};

	
	%Verbindungen defineren
	\draw [arrow](start) -- (pro1);
	\draw [arrow](pro1) -- (pro2);
	\draw [arrow](pro2) -- (ent1);
	\draw [arrow](ent1) -- (pro2-2);
	\draw [arrow](ent1) -- (pro3);
	\draw [arrow](ent1) -- node[anchor=east] {Ja} (pro3);
	\draw [arrow](ent1) -- node[anchor=south] {Nein} (pro2-2);
	\draw [arrow](pro2-2) |- (pro2);
	\draw [arrow](pro3) -- (pro4);
	\draw [arrow](pro4) -- (pro5);
	\draw [arrow](pro5) -- (ent2);
	\draw [arrow](ent2) -- (pro6);
	\draw [arrow](ent2) -- node[anchor=east] {Ja} (pro6);
	\draw [arrow](ent2) -- node[anchor=south] {Nein} ++(4,0) |- (pro5);
	%\draw [arrow](ent2) -- node[anchor=south] {Nein} (pro4);
	\draw [arrow](pro6) -- (pro7);
	\draw [arrow](pro7) -- (ende);
	\end{tikzpicture}
\end{document}